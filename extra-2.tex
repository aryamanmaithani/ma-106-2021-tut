\documentclass[12pt]{article}
\usepackage[lmargin=1in,rmargin=1in,tmargin=1in,bmargin=1in]{geometry}
\usepackage{amsmath, amssymb, amsfonts, amsthm, mathtools}
\usepackage[utf8]{inputenc}
\usepackage[inline]{enumitem}
\usepackage{cancel}
\usepackage{soul}
\usepackage[colorlinks=true]{hyperref}
\usepackage{datetime2}
\usepackage{centernot}

\setlength\parindent{0pt}
\let\emptyset\varnothing
\newcommand{\rank}{\operatorname{rank}}
%\renewcommand{\span}{\operatorname{span}}

\usepackage{xcolor}
\definecolor{mybgcolor}{RGB}{50, 50, 50} %46, 51, 63
\newcommand{\hint}[1]{\textbf{HIDDEN:} {\color[rgb]{0.95, 0.95, 0.95}#1}}

% \usepackage{pagecolor}
% \pagecolor{mybgcolor}
% \color{white}

\renewcommand{\familydefault}{\sfdefault}

% \usepackage{titlesec}
% \titleformat{\section}[block]
%   {\normalfont\scshape}{\S\thesection}{0.25cm}{\large}

% \usepackage{geometry}
% \geometry{
% 	a4paper,
% 	total={170mm,257mm},
% 	left=20mm,
% 	top=20mm,
% }

\title{(Extra)\texorpdfstring{$^2$}{2} Questions for MA 106}
\author{Aryaman Maithani}%\\
%\small TA for D1-T5}
\date{Semester: Spring 2021\\ Latest update: \DTMnow}

\begin{document}
\maketitle

These are questions that came out of some discussions.

\textbf{Notations}
\begin{enumerate}
	\item $\mathbb{F}$ denotes an arbitrary field. You may read \href{https://aryamanmaithani.github.io/ma-106-2021-tut/fields-and-vector-spaces.pdf}{this} to get an introduction to fields. Or may assume that $\mathbb{F} = \mathbb{R}$ or $\mathbb{C}.$ (Although your answers then may not work for a general field.)
	%
	\item Given a linear transformation $T,$ $\mathcal{N}(T)$ denotes the null space of $T.$
\end{enumerate}


\begin{enumerate}[leftmargin=*]
	\item A \textbf{nonempty} subset $J \subset \mathbb{F}^{n \times n}$ is said to be a \emph{two-sided ideal} if it has the following properties:
	\begin{enumerate}
		\item (Closed under addition) For all $A, B \in J,$ we have $A + B \in J,$
		\item (Absorption) For all $A \in J$ and $C \in \mathbb{F}^{n \times n},$ we have $AC, CA \in J.$
	\end{enumerate}
	Show that the (two-sided) ideals of $\mathbb{F}^{n \times n}$ are precisely $\{O\}$ and $\mathbb{F}^{n \times n}.$
	%
	\item Let $A \in \mathbb{F}^{n \times n}$ be such that $Ay = y$ for all $y \in \mathbb{F}^{n \times 1}.$ Show that $A = I.$\\
	\hint{Consider $y = e_k$ for $k \in \{1, \ldots, n\}.$}
	%
	\item Suppose $A \in \mathbb{R}^{2 \times 2}$ is such that $x^\top Ax = 0$ for all $x \in \mathbb{R}^{2 \times 1}.$ Is it necessary that $A = O?$\\
	\hint{No. Interpret $x^\top Ax$ as $\langle Ax, x\rangle.$}
	%
	\item Let $P \in \mathbb{R}^{n \times n}$ be invertible and let $A = P^{\top}P.$ \\
	Show that if $x \in \mathbb{R}^{n \times 1},$ then $x^\top Ax = 0 \iff x = 0.$
	%
	\item Let $A \in \mathbb{F}^{n \times n}$ be arbitrary. Show that
	\begin{enumerate}
		\item $A$ can be written as a sum of two invertible matrices, and
		\item $A$ can be written as a sum of two non-invertible matrices.
	\end{enumerate}
	%
	\item Can every matrix $A \in \mathbb{F}^{n \times n}$ be written as a product $LU$ where $L, U \in \mathbb{F}^{n \times n}$ are lower and upper triangular, respectively?\\
	\hint{No. Try to find a counterexample for $n = 2.$}
	%
	\item Let $V$ and $W$ be finite dimensional vector spaces over $\mathbb{F}.$ Let $T : V \to W$ be a linear transformation and let $B = \{v_1, \ldots, v_k\}$ be a basis of $\mathcal{N}(T).$ Extend $B$ to a basis $B' = B \cup \{v_{k + 1}, \ldots, v_n\}$ of $V.$ Show that $\{T(v_{k+1}), \ldots, T(v_n)\}$ is a basis of range (or image) of $T.$\\
	What can you say about the dimensions involved? Does this seem familiar?
	%
	\item Let $V$ and $W$ be finite dimensional vector spaces over $\mathbb{F}.$ Let $T : V \to W$ be a linear transformation and let $B = (v_1, \ldots, v_k)$ be an ordered basis of $\mathcal{N}(T).$ Extend $B$ to an ordered basis $B' = (v_1, \ldots, v_k, v_{k + 1}, \ldots, v_n)$ of $V.$ Fix any ordered basis $C$ of $W.$ What can you say about the first $k$ columns of $M_{B'}^C(T)?$ The last $n - k?$
	%
	\item Let $V$ be a finite  dimensional vector spaces over $\mathbb{F}$ and $T : V \to V$ a linear transformation. Show that there exist ordered bases $B$ and $C$ of $T$ such that the matrix $M_B^C(T)$ is diagonal with entries $1, \ldots, 1, 0, \ldots, 0.$
\end{enumerate}
\end{document}