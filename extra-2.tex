\documentclass[12pt]{article}
\usepackage[lmargin=1in,rmargin=1in,tmargin=1in,bmargin=1in]{geometry}
\usepackage{amsmath, amssymb, amsfonts, amsthm, mathtools}
\usepackage[utf8]{inputenc}
\usepackage[inline]{enumitem}
\usepackage{cancel}
\usepackage{soul}
\usepackage[colorlinks=true]{hyperref}
\usepackage{datetime2}
\usepackage{centernot}

\setlength\parindent{0pt}
\let\emptyset\varnothing
\newcommand{\rank}{\operatorname{rank}}
%\renewcommand{\span}{\operatorname{span}}

\usepackage{xcolor}
\definecolor{mybgcolor}{RGB}{50, 50, 50} %46, 51, 63

% \usepackage{pagecolor}
% \pagecolor{mybgcolor}
% \color{white}

\renewcommand{\familydefault}{\sfdefault}

% \usepackage{titlesec}
% \titleformat{\section}[block]
%   {\normalfont\scshape}{\S\thesection}{0.25cm}{\large}

% \usepackage{geometry}
% \geometry{
% 	a4paper,
% 	total={170mm,257mm},
% 	left=20mm,
% 	top=20mm,
% }

\title{(Extra)\texorpdfstring{$^2$}{2} Questions for MA 106}
\author{Aryaman Maithani}%\\
%\small TA for D1-T5}
\date{Semester: Spring 2021\\ Latest update: \DTMnow}

\begin{document}
\maketitle

These are questions that came out of some discussions.

\begin{enumerate}
	\item A \textbf{nonempty} subset $J \subset \mathbb{R}^{n \times n}$ is said to be a \emph{two-sided ideal} if it has the following properties:
	\begin{enumerate}
		\item (Closed under addition) For all $A, B \in J,$ we have $A + B \in J,$
		\item (Absorption) For all $A \in J$ and $C \in \mathbb{R}^{n \times n},$ we have $AC, CA \in J.$
	\end{enumerate}
	Show that the ideals of $\mathbb{R}^{n \times n}$ are precisely $\{O\}$ and $\mathbb{R}^{n \times n}.$
\end{enumerate}

\end{document}